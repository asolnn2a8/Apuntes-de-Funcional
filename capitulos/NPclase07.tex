%!TEX root = ../main.tex

%%%%% NÚMERO DE LA CÁTEDRA Y FECHA
\renewcommand{\catnum}{9 (7 No Presencial)}% 6NP %numero de catedra
\renewcommand{\fecha}{14 de abril de 2020}
%%%%%%%%%%%%%%%%%%
%Encabezado
\NAM[
\begin{minipage}{0.6\textwidth}
]{\begin{minipage}{0.7\textwidth}}
\begin{flushleft}
\hspace*{-0.5cm}\textbf{Fecha:} \fecha.
\end{flushleft}
\end{minipage}

\begin{center}
\LARGE\textbf{Cátedra \catnum}
\end{center}

%Fin encabezado

\section*{Operadores no Acotados y Noción de Operador Adjunto}

\begin{defn}
Una función $T: \underbrace{\cd(T)}_{\text{sev $E$}}  \subseteq E \to F$ lineal es un operador no acotado
\end{defn}

\begin{defn}
$T$ se dice acotado si $\exists c \geq 0 :\ \norm[F]{Tx} \leq c \norm[E]{x}, \quad \forall x \in \cd(T)$
\end{defn}

\begin{defn}
Un operador  $T: \cd(T)  \subseteq E \to F$ se dice \textbf{cerrado} si $\gr T$ es cerrado en $E \times F$, donde $\gr T = \{ (x,\ Tx) | x \in \cd(T) \}$
\end{defn}

\begin{note}: 
\begin{enumerate}[(1)]
    \item $T$ cerrado $\implies \ker T$  cerrado $$\ker T \times \{ 0 \} = \gr T \cap (E \times \{ 0 \})$$
    
    \item$T$ cerrado $\iff (x_n) \subseteq \cd(T)$
\end{enumerate}
$$y_n = T x_n , \quad 
\left.
x_n \to x \atop
y_n \to y
\right \} \implies x \in \cd(T) \ \wedge \ y = T x$$
\end{note}

\section{Noción de Adjunto}

Queremos generalizar la noción de matriz transpuesta, y se define el adjunto (o traspuesto) para esta generalización en dimensión infinita.

Sean $E$, $F$ ev Banach
$T : \cd(T) \subseteq E \to F$ tal que $\adh \cd(T) = E$
Queremos definir un operador $T\adj$ , adjunto de $T$ a partir de la propiedad fundamental (o básica) siguiente:

\begin{equation} \label{eqn:op-adj}
    \prodint[F,F']{Tx}{y} = \prodint[E,E']{x}{T\adj y}, \quad \forall x \in \cd(T),\ \forall y \in \cd(T\adj)
\end{equation}

Si uno quiere que \eqref{eqn:op-adj} sea cierto, lo primero que observamos es que $T\adj : \cd(T\adj) \subseteq F' \to E' $ y además $y \in \cd(T\adj) \iff $ \eqref{eqn:op-adj} se cumple $ \forall x \in \cd(T) \iff x \in \cd(T) \mapsto \prodint[F, F']{Tx}{y} $ es una forma lineal continua en $\cd(T)$

Así, $$ \cd(T\adj) = \{ y  \in F' | \exists c \geq 0 ,\ \abs{ \prodint{Tx}{y} } \leq c \norm[E]{x} \ \forall x \in \cd(T) \} $$

Además, si $y \in \cd(T\adj)$, entonces la forma lineal $g_y : \cd(T) \subseteq E \to \rr $ tal que $ x \mapsto \prodint{Tx}{y} $ es continua, y entonces se extiende de manera única a una forma lineal continua $\tilde{g_y} : E \to F$, además 

$$ \abs{\prodint{ \tilde{g_y} }{x}} \leq c \norm[E]{x} \quad \forall x \in E$$

Se define $$ T\adj y = \tilde{g_y} $$

\begin{prop}
$T: \cd(T) \subseteq E \to F,\ \adh T = E$ entonces $T\adj$ es cerrado
\end{prop}

\begin{proof}
Sea $(y_n, T\adj y_n) \in \gr T\adj \subseteq F' \times E'$ tal que $y_n \to z $ en $F' \wedge T\adj y_n \to u$  en $E'$

pdq $z \in \cd(T\adj) ,\ u = T\adj z $

Ahora bien sabemos que $\forall x \in \cd(T)$

$$ \prodint{Tx}{y_n} = \prodint{x}{T\adj y_n} $$

y haciendo $n \to \infty$

$$ \prodint{Tx}{z} = \prodint{x}{u} ,\ \forall x \in \cd(T) $$

y entonces $z \in \cd(T\adj)$, pues

$$ \abs{ \prodint{Tx}{y} } = \abs{ \prodint{x}{u} } \leq \norm[E']{u} \norm[E]{x} \leq c \norm[E]{u} ,\ \foral x \in \cd(T) $$

Además $T\adj z = u$
\end{proof}

Todo esto es un espiritu llamado ``la dualidad'', la cual es posiblemente la mayor contribución de las matemáticas a la humanidad. Los operadores duales son ocupadas ampliamente en la mecánica cuántica, matematizandola, y mejorando el entendimiento del mundo que nos rodeamos. Aunque estos conceptos no sean fácil de comprender.

En optimización, también se ocupa bastante los conceptos de dualidad.

\begin{note}

\begin{align*}
\text{Sea } &(y,\ T\adj y = f) \in \gr T\adj \\
&\iff \prodint[]{Tx}{y} = \prodint[]{x}{f} \ \forall x \in \cd(T) \\
&\iff -\prodint[]{x}{f} + \prodint[]{Tx}{y} = 0 \ \forall x \in \cd(T) \\
&\iff  \prodint[]{(x,\ Tx)}{(-f,\ y)}  = 0 \ \forall x \in \cd(T) \\
&\iff  (-f,\ y) \in (\gr T)\ort  \ \forall x \in \cd(T) 
\end{align*}

$J(\gr T\adj) = (\gr T)\ort $, donde $J: (y,\ f) \in  F' \times E' \mapsto (-f,\ y) \in E' \times F'$

\end{note}







% \begin{ex}:\\
% (iii) Sea $E$ Hilbert. Todo sev $G$ de $E$, cerrado, posee suplementario topológico y es: $$
% L = G\ort = \{ x \in E | \prodint{x}{y} = 0\ \forall y \in G \}
% $$
% (iv) $E$ Banach y $f_1, \ \ldots, \ f_N \in E'$ l.i. $b = \{ x \in E | \prodint{f_i}{x} = 0 \ \forall i = 1,\ \ldots ,\ N \}$ (ej sev cerrado de $E$)
% \end{ex}

% \begin{prop}
% $\codim b = N$
% \end{prop}

% \begin{proof}
% Veamos que $\exists x_1,\ \ldots ,\ x_n \in E$ l.i. tales que $E = G \oplus \gen[n]{x}$.

% Consideremos la aplicación $\varphi : E \to \rr[N]$ tal que $\varphi (x) = (\prodint{f_1}{x},\ \ldots ,\ \prodint{f_N}{x})$. $\varphi$ es sobreyectivo. En efecto, si no, $\exists \Vec{\alpha} \in \rr[N]$ tal que $\Vec{\alpha}$ es ortogonal a $(\prodint{f_1}{x},\ \ldots ,\ \prodint{f_N}{x}),\ \forall x \in E$ y entonces $ \sum_{i=1}^{n} \alpha_i f_i (x)  = 0  \implies \Ver{\alpha} = 0 $ por li, lo cual es una contradicción.

% Así, $\varphi$ es sobreyectivo. Tomando $e_i \in \rr[N] ,\ \exists x_i \in E$ tal que $\varphi (x_i) = e_i$, o sea $$ \prodint{f_j}{x_i} = \delta_i (j) \ \forall i,\ j \in \nn$$
% sea $x \in E$ y escribimos $$ x = \underbrace{\sum_{i=1}^{n} f_i(x) \cdot (x_i)}_{\in \gen{x}} + \underbrace{\left( x - \sum_{i=1}^{n} \prodint{f_i}{x} x_i \right)}_{\in G} $$ 

% Sea $j \in \cind{n}$, 
% \begin{align*}
%     \prodint{f_j}{ x - \sum_{i=1}^{n} \prodint{f_i}{x} \cdot x_i } &= \prodint{f_j}{x} - \sum_{i=1}^{n} \prodint{f_i}{x} \cdot \underbrace{\prodint{f_j}{x_i}}_{\delta_{i} (j)} \\
%     &=  \prodint{f_j}{x} - \prodint{f_j}{x} = 0
% \end{align*}
% \end{proof}

% (v) Sea $E$ Banach no isomorfo a un espacio de Hilbert. Siempre es posible encontrar sev de $E$ cerrado, que no poseen suplementario topológico.

% \begin{ex}
% $E = \ell^{\infty}$ y $G = \Cc_{0} = \{ (x_n)_n \in \ell^{\infty} | x_n \to 0 \}$
% \end{ex}

% \section*{Noción de Ortogonalidad en Espacios de Banach}

% \begin{defn}
% Sea $E$ Banach\\
% ($\cdot$) $M$ sev de $E$, se define 
% \end{defn}