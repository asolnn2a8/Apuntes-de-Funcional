%!TEX root = ../main.tex

%%%%% NÚMERO DE LA CÁTEDRA Y FECHA
\renewcommand{\catnum}{\theNPclase \ No Presencial}% 6NP %numero de catedra
\renewcommand{\fecha}{01 de junio de 2020}
%%%%%%%%%%%%%%%%%%
%Encabezado
\NAM[
\begin{minipage}{0.6\textwidth}
]{\begin{minipage}{0.7\textwidth}}
\begin{flushleft}
\hspace*{-0.5cm}\textbf{Fecha:} \fecha.
\end{flushleft}
\end{minipage}

\begin{center}
\LARGE\textbf{Cátedra \catnum}
\end{center}

%Fin encabezado

Módulo \#3

\begin{thm}[Banach-Alaogh-Bourboki]
$E$ espacio de Banach. Entonces
\begin{equation*}
    \overline{B}_{E'}(0, 1)\ \text{en $E'$ es compacta para la topología débil*}
\end{equation*}
\end{thm}

Este teorema es importante porque nos permite hallar compactos en el dual (recordando que la compacidad nos permite asegurar las existencias).

\begin{proof}
La idea consiste en identificar el espacio $[E',\, \topdebilE]$ con un sev de $[S = \rr^{E},\, \topprod]$ donde $\topprod}$ 
es la top. producto, ie, es la topología menos fina en $S$ que hace continuas las proyecciones
\begin{equation*}
    \omega = (\omega_{y})_{y\in E} \overset{\phi_{x}}{\longmapsto} \omega_x, \quad \forall x \in E
\end{equation*}

Esta identificación lo haremos a través de la función
\begin{equation*}
    \deffunc{\Phi}
    { [E',\, \topdebilestrE] }{ [S,\, \topprod] }
    {f \in E'}{\big( \prodint[]{f}{y} \big)_{y \in E} \in S }
\end{equation*}

Probemos que $\Phi$ es continua y que $\Phi$ es biyectiva de $E'$ en $\Phi(E')$ y que $\Phi^{-1}$ es también continua.

Con esto, tendríamos que $\Phi$ se identifica homeomórficamente con $\Phi(E')$. 

Admitiendo esto, probemos que $\Phi(\overline{B}_{E'} (0; 1)) = K$ es compacto en $[S,\, \topprod]$. En efecto,
\begin{eqnarray}
    K &=& \conj{
    \omega = (\omega_x)_{x\in E} \big|
    \abs{\omega_{x}} \leq \norm[E']{x},\ \omega_{x+y} = \omega_{x} + \omega_{y},\ \omega_{\lambda x} = \lambda \omega_{x},\, \forall x,y \in E,\ \forall \lambda \in \rr
    } \\
    &=& K_1 \cap K_2
\end{eqnarray}
con 
\begin{equation}
    K_1 = \conj{
    \omega\in S \big|
    \abs{\omega_{x}} \leq \norm[E']{x} \forall x \in E
    }
\end{equation}
y
\begin{equation}
    K_2 = \conj{
    \omega \in S \big|
    \omega_{x+y} = \omega_{x} + \omega_{y},\ \omega_{\lambda x} = \lambda \omega_{x},\, \forall x,y \in E,\ \forall \lambda \in \rr
    }
\end{equation}

Pero,
\begin{equation}
    K_1 = \prod_{x \in E} \big[ -\norm[]{x},\, \norm[]{x} \big]
\end{equation}

Y entonces $K_1$ es un producto de compactos, y así, $K_1$ es compacto.

Por otro lado, $K_2$ es cerrado, pues las funciones $\omega \mapsto \omega_{x+y} - \omega_x - \omega_y$ y $\omega \mapsto \omega_{\lambdax} - \lambda \omega_x $ son continuas $\forall x,\, y \in E$ $\forall \lambda \i $. Luego, los conjuntos
\begin{eqnarray}
    A_{x,y} &=& \conj{
    \omega \in S| \omega_{x+y} - \omega_x - \omega_y
    }\\
    B_{\lambda x} &=& \conj{
    \omega \in S | \omega_{\lambda x} - \lambda \omega_x
    }
\end{eqnarray}

Son cerrados en tanto que preimágenes del singletón $\conj{0}$

(Continuación del módulo \#3)

\begin{equation*}
    \deffunc{\Phi}
    { [E',\, \topdebilestrE] }{ [S,\, \topprod] }
    {f \in E'}{\big( \prodint[]{f}{y} \big)_{y \in E} \in S }
\end{equation*}

\begin{itemize}
    \item Demostremos que $\Phi$ es inyectiva. Esto es pues si $\Phi(f) = 0 \implies f = 0$.
    
    Así, $\Phi$ es una biyección de $E'$ en $\Phi(E')$.
    
    \item $\Phi$ es continua. Para probarlo, usaremos la proposición \ref{prop:prop-1-top-debil} de los preliminares, que nos permite afirmar que $\Phi$ es continua $\iff$ $p_{x}\circ \Phi  $ es continua $\forall x \in E$, y donde \begin{equation*}
        \deffunc{p_x}
        {S}{\rr}
        {(\omega_{y})_{y\in E}}{\omega_{x}}
    \end{equation*}
    
    Pero, en nuestro caso,
    \begin{equation*}
        \deffunc{p_{x} \circ \Phi}
        {[E',\, \topdebilestrE]}{\rr}
        {f}{\prodint[]{f}{x} = \prodint[]{Jx}{f}}
    \end{equation*}
    
    que es continua por definición de la topología débil* $\topdebilestrE$.
    
    \item 
    \begin{equation*}
        \deffunc{\Phi\inv}
        {[\Phi(E'),\, \topprod]}{[E',\, \topdebilestrE]}
        {\omega = (\prodint[]{f}{y})_{y \in E}}{f}
    \end{equation*}
    % $\deffuncs{\Phi\inv}{[\Phi(E'),\, \topprod]}{[E',\, \topdebilestrE]}$
    
    % $\omega = (\omega_y)_{y \in E} = (\prodint[]{f}{y})_{y \in E} \mapsto f$
    
    Usando la misma proposición, ahora $\Phi\inv$ es continua $\iff$ $Jx \circ \Phi\inv$ es continua, $\forall x \in E$
    
    Pero en nuestro caso, $\forall x \in E$, 
    \begin{equation*}
        \deffunc{J_x \circ \Phi\inv}
        {[\Phi(E'),\, \topprod]}{\rr}
        {(\prodint[]{f}{y})_{y\in E}}{\prodint[]{Jx}{f}=\prodint[]{f}{x} = p_x \big( (\prodint[]{f}{y})_{y\in E} \big)}
    \end{equation*}
    
    ie, $J_x\circ \Phi\inv = p_x$, que es continua por definición de la topología producto.
\end{itemize}

\end{proof}

Módulo \#4

\section{Espacios reflexivos}

\begin{defn}[Espacio reflexivo]
$E$ esp. de Banach. $E$ se dice reflexivo si $J(E) = E''$. ie, cuando $J$ es sobreyectiva
\end{defn}

\begin{ex}:\\
\begin{ienumerate}
    \item $\dim E < + \infty \implies J$ es sobre, pues es inyectiva.
    \item $E = \ell_p$ con $1 < p < \infty$.
    \item $E = \ell_1$ o $E = \ell_\infty$, no son reflexivos.
\end{ienumerate}
\end{ex}

\begin{note}:\\
\begin{ienumerate}
    \item Si $E$ es reflexivo, entonces $\topdebil{E'}{E''}$ (topología débil en $E'$) y $\topdebilestrE$ (topología débil* en $E'$) coinciden.
    
    \item Supongamos que $E$ es reflexivo, y entonces, salvo identificación, $E = (E')'$ (ie, es un dual), y le podemos aplicar (en principio) el teorema de B-A-B, ¿Cómo hacerlo, y qué conclusión sacamos? El problema es que $E$ no es un dual, si no que \textbf{se identifica} por un dual, así que hay que trabajarlo.
    
\end{ienumerate}

\end{note}

Para responder, consideremos el esquema siguiente:
\begin{equation*}
    \deffuncs{J}
    {E}{[E'',\, \topdebil{E''}{E'}]}
\end{equation*}

Por B-A-B, $\overline{B}_{E''}(0,1)$ es compacta en $E''$ para $\topdebil{E''}{E'}$, queremos entonces saber con qué topología en $E$, $J\inv$ es continua, para concluir compacidad de $\overline{B}_{E}(0,1) = J\inv (\overline{B}_{E''}(0,1))$ (tenemos continuidad para $J\inv$)

Muy fácil, miremos el esquema siguientea la luz de la prop. \ref{prop:prop-1-top-debil} de los preliminares:
\begin{equation*}
    \deffuncs{J\inv}
    {[E'',\, \topdebil{E''}{E'}]}{[E,\, \topdebilE]}
\end{equation*}

Fijamos $f \in E'$ y componemos con $f$. Luego decimos que $J\inv$ es continua $\iff$ $f \circ J\inv$ es continua $\forall f \in E'$. 

Pero,
\begin{equation*}
    \deffunc{f \circ J \inv}
    {[E'',\, \topdebil{E''}{E'}]}{\rr}
    {\xi = Jx}{x \mapsto \prodint[]{f}{x} = \prodint[]{Jx}{f}
    }
\end{equation*}

Que es continua para cada $f \in E'$, pues
\begin{equation*}
    \abs{
    \prodint[]{Jx}{f}
    } 
    \leq \norm[]{f} \norm[]{Jx} = C \norm[]{\xi}
\end{equation*}

Concluimos que $J\inv$ es continua al dotar $E$ de la topología débil, $\topdebilE$. Así, B-A-B, nos dice que $\overline{B}_E (0,1) = J\inv (\overline{B}_{E''} (0,1))$ es compacta para $\topdebilE$.

Esto presenta una implicación en:
\begin{thm}
$E$ Banach.
\begin{equation*}
    E \text{ reflexivo} \iff \overline{B}_E(0, 1) \text{ es débilmenente compacta}
\end{equation*}
\end{thm}

\begin{proof}
La implicancia ($\Leftarrow$) es técnicamente tediosa (ver libro Brezis)
\end{proof}

\begin{prop}
Sea $E$ Banach reflexivo y $M$ sev de $E$ cerrado. Luego $M$ dotado de la topología inducida por $E$, es reflexivo.
\end{prop}

\begin{proof}
Más adelante
\end{proof}

\begin{cor}\label{cor:corolario-1}
$E$ Banach
\begin{equation*}
    E \text{ reflexivo} \iff E' \text{ reflexivo}
\end{equation*}
\end{cor}

\begin{cor}\label{cor:corolario-2}
$E$ Banach reflexivo y $C$ subconjunto acotado, cerrado (fuerte) y convexo de $E$. Luego, $C$ es compacto débil de $E$.
\end{cor}














%!TEX root = ../main.tex

%%%%% NÚMERO DE LA CÁTEDRA Y FECHA
\renewcommand{\catnum}{\theNPclase \ No Presencial}% 6NP %numero de catedra
\renewcommand{\fecha}{01 de junio de 2020}
%%%%%%%%%%%%%%%%%%
%Encabezado
\NAM[
\begin{minipage}{0.6\textwidth}
]{\begin{minipage}{0.7\textwidth}}
\begin{flushleft}
\hspace*{-0.5cm}\textbf{Fecha:} \fecha.
\end{flushleft}
\end{minipage}

\begin{center}
\LARGE\textbf{Cátedra \catnum}
\end{center}

%Fin encabezado

¿Cómo aplicamos el Teo. de BAB a $E$?

Recordemos el siguiente teorema:
\begin{thm}\label{teo1:Erefl-Edual-refl}
$E$ Banach
\begin{equation*}
    E \text{ reflexivo} \iff \overline{B}_E(0, 1) \text{ es débilmenente compacta}
\end{equation*}
\end{thm}

\begin{prop}\label{prop:prop1}
Sea $E$ Banach reflexivo y $M$ sev cerrado fuerte de $E$. Entonces $M$, dotado de la topología inducida por $E$ es reflexivo.
\end{prop}

\begin{cor}\label{coro:corolario1-E-refl-Edual-refl}
$E$ Banach. $E$ es reflexivo $\iff$ $E'$ es reflexivo.
\end{cor}

\begin{cor}\label{coro:corolario2}
Sea $E$ Banach reflexivo y $K \subseteq E$ cerrado, acotado y convexo. Entonces $K$ es compacto débil en $E$.
\end{cor}

\begin{proof}[Demostración del Corolario \ref{coro:corolario1-E-refl-Edual-refl}]
(->) pdq $\overline{B}_{E'}(0, 1)$ es compacta para $\topdebil{E'}{E''}$

Pero sabemos que $\overline{B}_{E'}(0, 1)$ es compacta para $\topdebil{E'}{E = J(E)}$, esto es, para la topología débil* (BAB). Ahora bien, $E$ es reflexivo, entonces $J(E) = E''$, y entonces $\topdebil{E'}{E} = \topdebil{E'}{E''}$ (top. débil y débil* en $E'$, coinciden)

Así, $\overline{B}_{E'}(0, 1)$ es también débil compacta.

(<-) Vamos a suponer que $E'$ es reflexivo $\implies$ $E''$ es reflexivo $ \overset{\text{prop \ref{prop:prop1}}}{\implies} $ $J(E)$ (que es sev cerrado de $E''$) es reflexivo

Ahora bien, $[E,\, \norm[E]{\bullet}]$ es isométrico a $[J(E),\, \norm[E'']{\bullet}]$ y entonces (ejercicio: probar que si dos espacios de Banach son isométricos, entonces uno es reflexivo ssi el otro lo es) $E$ es reflexivo.

Otra forma de ver que $E$ es reflexivo es notar que 
\begin{equation}
    J\inv (\underbrace{\overline{B}_{E''}(0, 1)}_{\text{compacta débil}}) \supseteq \underbrace{\overline{B}_{E}(0, 1)}_{\text{cerrada débil}}
\end{equation}

$J\inv$ continua para $\topdebilE$. y entonces $J\inv ({\overline{B}_{E''}(0, 1)})$ es compacto débil en $E$, y así, ${\overline{B}_{E}(0, 1)}$ es compacto débil.

\end{proof}

\begin{proof}[Demostración del Corolario \ref{coro:corolario2}]
$K$ cerrado. acotado, convexo de $E$, reflexivo.

$K$ es cerrado débil por ser convexo y cerrado fuerte. Como $K$ es acotado, $\exists R > 0$, $K\subseteq \overline{B}_{E}(0, R)$, pero, $E$ reflexivo $\implies $ $\bolacerradaf[E]{0}{R}$ es compacta débil (por Teo. \ref{teo1:Erefl-Edual-refl}), y entonces $K$ es un cerrado débil dentro de un compacto débil, y así es compacto débil.
\end{proof}

\begin{proof}[Demostración de la Prop. \ref{prop:prop1}]
En $M$ podemos poner dos topologías ``débiles''; 
\begin{ienumerate}
    \item $\topdebil{M}{M'}$
    \item La topología inducida por $\topdebilE$, ie, las trazas sobre $M$ de la topología de los abiertos débiles de $E$.
\end{ienumerate}

\begin{exer}
Pruebe que \iitem[i] y \iitem[ii] coinciden, jugando con restricciones y extensiones de formas lineales sobre $E$ y  $M$.
\end{exer}

Con esto, como $E$ es reflexivo, $\bolacerradaunitariaf[E]$ es compacta para la topología débil (Teo. \ref{teo1:Erefl-Edual-refl}). Ahora bien, 
\begin{equation}
    \bolacerradaunitariaf[M] = \bolacerradaunitariaf[E] \cap M
\end{equation}

Donde $M$ es cerrado débil en $E$.
Luego, $\bolacerradaunitariaf[M]$ es compacto para la topología $\topdebilestrE$ y como esta coincide con $\topdebil{M}{M'}$, al tomar trazas, también es compacto para $\topdebil{M}{M'}$. Así, $M$ es reflexivo.
\end{proof}

Una pregunta natural es, si $(x_n)_n$ es sucesión acotada en un espacio $E$ \textit{reflexivo}, ¿es posible extraer subsucesión $(x_{n_k})_k$ convergente débil?

La sucesión $x_n$, por ser acotada, la podemos sumergir en una bola $\bolacerradaf[E]{0}{R}$, para algún $R$, que sabemos es compacta débil, pues $E$ es reflexivo.

Ahora bien, esto no es suficiente para poder extraer $(x_{n_k})_k$, convergente débil, pues $[E,\, \topdebilE]$ no es metrizable.

Sin embargo, aún hay esperanza de que la respuesta sea positiva, pues, bastaría con que el espacio topológico compacto $[\bolacerradaf[E]{0}{R},\, \topdebilE \cap \bolacerradaf[E]{0}{R}]$  fuese metrizable para poder usar Bolzano-Weiertrass y que la respuesta fuese sí.

\section{Espacios Separables}

\begin{defn}
$E$ Banach. $E$ se dice \textit{separable} ssi $\exists$ subconjunto de $E$ numerable denso.
\end{defn}

Recordar que no hay que confundir la definición de \textit{separable} con \textit{separado} (esto último, significa que sea Hausdorff).

\begin{thm} \label{teo1:Esp-Sep}
Sea $E$ Banach
\begin{ienumerate}
    \item $E$ es separable $\iff$ $\bolacerradaunitariaf[E']$ para $\topdebilestrE$ (débil*) es metrizable.
    \item $E'$ es separable $\iff$ $\bolacerradaunitariaf[E]$ para $\topdebilE$ es metrizable.
\end{ienumerate}
\end{thm}

Es decir, para saber que un espacio es separable, hay que mirar la bola unitaria primal.

\begin{thm}\label{teo2:Esp-Sep}
$E$ es Banach. 
\begin{equation}
    \text{$E$ es separable y reflexivo} \iff \text{E' es separable y reflexivo}
\end{equation}
\end{thm}

Hay que hacer notar el \textbf{``y''} entre medio, pues este teorema es válido solo para espacios reflexivos.

Con estas herramientas, regresemos a la pregunta natural