%!TEX root = ../main.tex

%%%%% NÚMERO DE LA CÁTEDRA Y FECHA
\renewcommand{\catnum}{\theNPclase \ No Presencial}% 6NP %numero de catedra
\renewcommand{\fecha}{27 de abril de 2020}
%%%%%%%%%%%%%%%%%%
%Encabezado
\NAM[
\begin{minipage}{0.6\textwidth}
]{\begin{minipage}{0.7\textwidth}}
\begin{flushleft}
\hspace*{-0.5cm}\textbf{Fecha:} \fecha.
\end{flushleft}
\end{minipage}

\begin{center}
\LARGE\textbf{Cátedra \catnum}
\end{center}

%Fin encabezado

\begin{thm}
$E$, $F$ esp. de Banach
$T : \cd(T) \subseteq E \to F$

$\adh \cd(T) = E$, $T$ es cerrado.

Son equivalentes:

\begin{ienumerate}
    \item $\cd(T) = E$
    \item $T$ es acot.
    \item $ \cd (T\adj) = F' $
    \item $T\adj$ es acotado
\end{ienumerate}

Bajo estas condiciones:$$
\norm[\cl (E;\ F)]{T} = \norm[\cl (F';\ E')]{T\adj}
$$

\end{thm}

\begin{proof}:\\

\iitem[i] $\implies$ \iitem[ii] Basta con usar el teorema del grafo cerrado.

\iitem[ii] $\implies$ \iitem[iii] Basta con usar la definición $\cd(T\adj)$.

\iitem[iii] $\implies$ \iitem[iv] Basta usar nuevamente Teo. Graf. Cerrado.

\iitem[iv] $\implies$ \iitem[i] (Este es un poco menos trivial) 

Comencemos probando que \iitem[iv] $\implies$ $\cd(T\adj)$ es cerrado. 
En efecto:

Sea $(y_n) \in \cd(T\adj)$ tq $y_n \to y \in F'$ , y probemos que $y \in \cd(T\adj)$,

La idea es usar la hipótesis \iitem[iv].

Como $T\adj$ es acot.,

% $$
% \norm[]{T\adj (y_n - y_m) 
% \leq \underbrace{\norm[]{T\adj}}_{< \infty} \norm[]{y_n - y_m}
% $$

Y entonces $(T\adj y_n)_n$ es de Cauchy en $E'$, y entonces converge, digamos,

$T\adj y_n \overset{n \to \infty}{\to} f \in E'$. Con esto, $((y_n,\ T\adj y_n))$ es convergente en $F' \times E'$, y como está en $\gr T\adj$, que es cerrado, se concluye que $(y,\ f) \in \gr T\adj$

En particular $y \in \cd(T\adj)$

Probemos ahora que $\cd(T)$ es también cerrado, lo que permite concluir $\cd(T) = E$ , pues $\cd(T)$ es denso en $E$. Para ello definamos $$
G = \gr T \wedge L = \{ 0 \} \times F
$$ sev de $E \times F$, cerrados, se tiene que $$
G + L = \cd(T) \times F
$$ 

$$
\left.
\begin{array}{r}
     G\ort = (\gr T)\ort = J(\gr T\adj )\\
     L\ort = E' \times \{ 0 \}
\end{array} \right \} G\ort + L\ort = E' \times \cd(T\adj)
$$

y entonces $G\ort + L\ort$ es cerrado en $E' \times F'$.

Luego, gracias al Teorema de la pág. 2 de la guía \#2, $G + L$ es cerrado, y entonces $\cd(T)$ es cerrado en $E$.

Por último, probemos que $\norm[]{T} = \norm[]{T\adj}$

Sabemos que $$
\prodint[]{y}{Tx} = \prodint[]{T\adj y}{x} \forall x \in \cd(T) \forall y \in \cd(T\adj)
$$

$$
\abs{ \prodint[]{y}{Tx} } \leq \norm[]{T\adj y} \norm[]{x} \leq \norm[]{T\adj} \norm[]{y} \norm[]{x} $$

$$
\underbrace{\sup_{y \in F'} \frac{ \abs{ \prodint[]{y}{Tx} } }{ \norm[]{y} }}_{= \norm[F]{T x}}
\leq \norm[]{T\adj} \norm[]{ x}
$$ y así $$
\norm[]{Tx} \leq \norm[]{T\adj} \norm[]{x}
$$ lo que implica que $
\norm[]{T} \leq \norm[]{T\adj}
$

$$
\norm[E']{ T\adj y } = \sup_{x \in E} \frac{
\abs{ \prodint[]{T\adj y}{x} }
}{
\norm[E]{x}
}
= \sup_{x \in E} \frac{
\abs{ \prodint[]{y}{T x} }
}{
\norm[E]{x}
}
\leq \frac{
\norm[]{y} \norm[]{Tx}
}{
\norm[]{x}
}
\leq \norm[]{y} \norm[]{T}
$$ y entonces $\norm[]{T\adj} \leq \norm[]{T}$

\end{proof}

\textbf{Notas y comentarios}

\begin{enumerate}[(1)]
    \item El teo. 2 posee una versión dual que dice que Si $T$ es a dominio denso y cerrado, entonces $T \adj $ es sobrey. $\iff$ $\exists c \geq 0:\ 
    \norm[]{x} 
    \leq c \norm[]{Tx} \quad \forall x \in \cd(T) \iff T \text{ es inyectivo}
    $ y $\im T$ es cerrada.
    
    Su dem. es una aplicación del Teo. 2 a $T\adj$, pues no sabemos que $T\adj$ tenga dominio denso.
    % $T\adj$ es sobrey. 
    \item Si $\dim E < + \infty$ o si $\dim F < +\infty$ entonces se tiene $$
    T \text{ es iny. } \iff T\adj \text{ es sobrey. }
    $$ $$
    T\adj \text{ es iny. } \iff T \text{ es sobrey. }
    $$ pues $\im T$ y $\im T\adj$ son cerrados, y
    como $
    (\ker T)\ort = \adh(\im T\adj) = E'
    $, y $
    \ker T = \{ 0 \}
    $. También $
    (\ker T\adj )\ort = \adh( \im T ) = F
    $
    % entonces gracias al Teo. 2 y su 
    
    En dimensión infinita, sólo se tiene $$
    T \text{ es sobrey} \implies T\adj \text{ es iny.}
    $$ $$
    T\adj \text{ es sobrey} \implies T \text{ es iny.}
    $$ Y la inversa \textbf{no} es cierta en gral.
    
    \textbf{Contraejemplo}
    
    $E = F = \ell_{2}$
    $T : E \to F$
    $u = (u_n)$
    $T u = v = \left( \frac{u_n}{n} \right)
    $
    
    Como $E$ es Hilbert, $E' = F' = \ell_{2}$. Además, $T$ es continua y que $T\adj = T$. En efecto, $$
    \prodint[]{Tu}{v} 
    = \prodint[]{ \left( \frac{u_n}{n} \right) }{ (v_n) } 
    = \sum_{n} \frac{u_n}{n} v_n
    = \sum_n u_n \frac{v_n}{n}
    = \prodint[]{u}{T\adj v}
    $$ Luego, $T\adj v = \left( \frac{v_n}{n} \right) = T v$
    
    Claramente, $T$ y $T\adj$ son inyectivos, pues $$
    T v = 0 
    \iff u \equiv 0
    $$ Pero, $T$ (resp. $T\adj$) \textbf{no} es sobre, pues $$
    \im T = \{ v \in \ell_2 | v = \left( \frac{u_n}{n} \right),\ u \in \ell_2 \} 
    % \varsubsetneqq \ell_2
    = \{ v \in \ell_2 | (n v_n) \in \ell_2 \}
    $$ $\left( \frac{1}{n^2} \right) \in \ell_2$, pero, $\left( n \frac{1}{n^2} \right) = \left( \frac{1}{n} \right) \not \in \ell_2$
    
    $\left( \frac{1}{n^2} \right) \in \ell_2\text{, pero, } \left( \frac{1}{n^2} \right) \not \in \im T$.
    % Pero, $\im T$ es cerrada, pues $$
    % \adh( \im T ) = ( \ker T\adj )\ort = ( \ker T )\ort
    % $$
    
\end{enumerate}

\newpage
\section{Doble adjunto ${T\dadj} $}

\begin{thm}
Sean $E$, $F$ Banach reflexivos. (ie, $E'' = E$, $F'' = F$). Si $T : \cd(T) \subseteq E \to F$ es cerrado y $\adh \cd(T) = E$, entonces $\cd(T\adj)$ es denso en $F'$.
\end{thm}

En este caso, podemos definir $T\dadj : \cd(T\dadj) \subseteq E'' \to F''$ o bien $T\dadj : \cd(T\dadj) \subseteq E \to F$ y $T \dadj = T$.