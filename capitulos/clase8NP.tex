%!TEX root = ../main.tex

%%%%% NÚMERO DE LA CÁTEDRA Y FECHA
\renewcommand{\catnum}{10 (8 No Presencial)}% 6NP %numero de catedra
\renewcommand{\fecha}{20 de abril de 2020}
%%%%%%%%%%%%%%%%%%
%Encabezado
\NAM[
\begin{minipage}{0.6\textwidth}
]{\begin{minipage}{0.7\textwidth}}
\begin{flushleft}
\hspace*{-0.5cm}\textbf{Fecha:} \fecha.
\end{flushleft}
\end{minipage}

\begin{center}
\LARGE\textbf{Cátedra \catnum}
\end{center}

%Fin encabezado

$T: D(T) \subseteq E \to F$, donde $\adh D(T) = E$ (se puede obviar pues es spg)

Se quiere definir $T\adj$, el operador adjunto

Que satisfaga la siguiente identidad

\begin{equation}\label{eqn:t-adj}
\prodint[F,F']{Tx}{y} = \prodint[E,E']{x}{T\adj y} \quad \forall x \in D(T) \ \forall y \in D(T\adj) 
\end{equation}

% $$ 
% \prodint[F,F']{Tx}{y} = \prodint[E,E']{x}{T\adj y} \quad \forall x \in D(T) \ \forall y \in D(T\adj) (*)
% $$

$$
\eqref{eqn:t-adj} \implies
\left\{ \begin{array}{l}
     T\adj: D(T\adj) \subseteq F' \to E' \\
D(T\adj) = \{ y \in F' |  x \mapsto \prodint[]{Tx}{y} ,\ x \in D(T) \text{ es lineal ie, $\exists c \geq o$; continua, $\abs{ \prodint[]{Tx}{y} } \leq x \norm[]{x} \ \forall x \in D(T) $ } \} \\
T\adj y = (x \in D(T) \to \prodint[]{Tx}{y})
\end{array}
% \left\{
% T\adj: D(T\adj) \subseteq F' \to E' \atop
% D(T\adj) = \{ y \in F' |  x \mapsto \prodint[]{Tx}{y} ,\ x \in D(T) \text{ es lineal ie, $\exists c \geq o$; continua, $\abs{ \prodint[]{Tx}{y} } \leq x \norm[]{x} \ \forall x \in D(T) $ } \} \atop
% T\adj y = (x \in D(T) \to \prodint[]{Tx}{y})
% \right.
$$

\begin{prop}:\\
(1) $T\adj$ es cerrado.

(2) $(\gr T)\ort = J( \gr T\adj )$ 
$$
\begin{array}{llll}
     J: &F' \times E' &\to &E' \times F' \\
     &(y,\ f) &\mapsto &(-f,\ y)
\end{array}
$$
\end{prop}

\begin{prop}
$T : D(T) \subseteq E \to F$ cerrado y $\adh D(T) = E$ entonces

(i) $
\ker T = (\im T\adj) \ort
$

(ii) $
\ker T\adj = (\im T) \ort
$

(iii) $
(\ker T)\ort \supseteq \adh (\im T\adj) 
$

(iv) $
(\ker T\adj)\ort = \adh (\im T)
$
\end{prop}

\begin{proof}:\\
$G \defeq \gr T$, $L \defeq E \times \{ 0 \}$. $G$, $L$ sev de $E \times F$, ¡cerrado!
\end{proof}

\begin{lem}\label{lem:ker-img}
$T: D(T) \subseteq E \to F$ no-acotado

y sean $G = \gr T \wedge L = E \times \{ 0 \}$. Entonces

(i) $
\ker T \times \{ 0 \} = G \cap L
$

(ii) $
E \times \im T = G + L
$

(iii) $
\{ 0 \} \times \ker T\adj = G\ort \cap L\ort
$

(iv) $
\im T\adj \times F' = G\ort + L\ort
$
\end{lem}

\begin{proof}
\textbf{Queda como ejercicio para el lector c:}
\end{proof}

\begin{proof}(De la proposición) \\

(i) pdq $$\ker T = (\im T\adj)\ort$$

Gracias a (iv) en el lema \ref{lem:ker-img}, y tomando $\perp$ $$
(\im T\adj \times F')\ort = (G\ort + L\ort)\ort = G \cap L
$$ y entonces $$
(\im T\adj)\ort \times \{ 0 \} = G cap L = \ker T \times \{ 0 \}
$$ Luego, $(\im T\adj)\ort = \ker T$.

(ii) pdq $$
\ker T\adj = (\im T )\ort
$$

Gracias a (ii) en el lema \ref{lem:ker-img} y tomando $\perp$ nos queda que: $$
(\im T \times F)\ort = (G + L)\ort \overset{\text{ver guia 2}}{=} G\ort \cap L\ort
$$ y entonces $$
 \{ 0 \} \times (\im T )\ort = G\ort \cap L\ort \overset{\text{lema \ref{lem:ker-img}}}{=} \{ 0 \} \times \ker T \adj
$$

(iii) pdq $$
(\ker T)\ort \supseteq \adh( \im T\adj )
$$

Tomando $\perp$ en (i), $$
(\ker T)\ort = {(\im T\adj)\ort}\ort \supseteq \adh(\im T \adj)
$$

(iv) Basta toma $\perp$ a (ii)
\end{proof}

\begin{note}
Hay ejemplos explícitos en que la inclusión en (iii) es estricta.

Resolver el ejercicio \#2, lista \#3:

$$
E = \ell_1 (\implies E' = \ell_{\infty})
$$
$$
\begin{array}{rcl}
     T : \ell_1 &\to &\ell_{\infty} \\
    %  tq 
     u=(u_n)_n &\mapsto &Tu = \left( \frac{u_n}{n} \right)_n
\end{array}
% T: \ell_1 \to \ell_{\infty} \ tq \ u=(u_n)_n \mapsto Tu = \left( \frac{u_n}{n} \right)_n
$$

Calcular $\ker T$, $(\ker T )\ort $, $T\adj$, $\ker T\adj$, $(\ker T\adj)\ort$, $\im T$, $\im T\adj$, etc.
\end{note}

\begin{exer}[Caracterización de operadores a imagen cerrada]
% \underline{\textbf{Tarea \# 1}} (Caracterización de operadores a imagen cerrada)

\textbf{Nuestro sueño:}
\begin{center}
$ \ulcorner \quad \quad
T u = f
\quad \quad \lrcorner $    
\end{center}

Sea $T: D(T) \subseteq E \to F$, $\adh D(T) = E$, $T$ cerrado

\textbf{TFAE:}

(i) $\im T$ es cerrada en $F$

(ii) $\im T\adj$ es cerrada en $E'$

(iii) $(\ker T)\ort = \im T\adj$

(iv) $(\ker T\adj)\ort = \im T$

\end{exer}

\begin{thm}[Caracterización de operadores sobreyectivos]: \\

Bajo las mismas hipótesis del ejercicio anterior

TFAE

(i) $T$ es sobreyectivo

(ii) $\exists c \geq 0:\ \norm[F']{y} \leq c \norm[E']{T\adj y} \ \forall y \in D(T\adj) $

(iii) $T\adj$ es inyectivo $( \ker T\adj = \{ 0 \} )$ $\wedge $ $\im T\adj$ es cerrada.
\end{thm}

\begin{note}
Cómo se usa la equivalencia (i)$\iff$(ii) en la práctica. La idea es plantear la ecuación $$
T\adj v = g \quad \text{(en $E'$)}
$$

En la práctica los operadores son autoadjuntos o simétricos, por lo tanto esta ecuación es prácticamente la misma que la original $(T u = f)$

Entonces nos planteamos esta ecuación con $g$ cualquiera y se dice así mismo, siempre que $v$ tiene solución y trato de probar que $\norm[]{v} \leq c \norm[]{g}$ con $c$ indep. de la función $v$. 

Entonces tratamos de estimar la solución en vez de encontrarla.

Esto se llama la \textbf{Técnica de las estimaciones a priori}.
\end{note}