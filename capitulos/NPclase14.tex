%!TEX root = ../main.tex

%%%%% NÚMERO DE LA CÁTEDRA Y FECHA
\renewcommand{\catnum}{\theNPclase \ No Presencial}% 6NP %numero de catedra
\renewcommand{\fecha}{26 de mayo de 2020}
%%%%%%%%%%%%%%%%%%
%Encabezado
\NAM[
\begin{minipage}{0.6\textwidth}
]{\begin{minipage}{0.7\textwidth}}
\begin{flushleft}
\hspace*{-0.5cm}\textbf{Fecha:} \fecha.
\end{flushleft}
\end{minipage}

\begin{center}
\LARGE\textbf{Cátedra \catnum}
\end{center}

%Fin encabezado

Módulo \#1

\begin{itemize}
    \item Caracterización de cerrado fuertes que también son cerrados débiles
    
    \item Funciones lineales continuas fuertes son también continuas débiles
    
    \item Noción de espacio bidual
\end{itemize}

\begin{thm}\label{teo:teo-1-C-cerrado-debil}
Sea $C$ un convexo de un espacio de Banach $E$. Luego,
\begin{equation*}
    \text{ $C$ es cerrado fuerte $\iff$ $C$ es cerrado débil } 
\end{equation*}
\end{thm}

\begin{proof}:\\
\begin{itemize}
    \item[($\Leftarrow$)] Fácil pues $\topdebilE \subseteq \topDeLaNorma$
    \item[($\Rightarrow$)] Sea $C$ cerrado fuerte y convexo.
    
    Sea $x_0 \in \compl{E} C$. como $x \not \in C$, podemos separar estrictamente $\conj{x_0}$ de $C$. Luego $\exists f \in E', \ \exists \alpha \in \rr$ tal que
    \begin{equation*}
        \prodint[]{f}{x} < \alpha < \prodint[]{f}{x_0} \quad \forall x \in C
    \end{equation*}
    
    Entonces 
    \begin{eqnarray}
        V &=& \conj{
        x \in E | \prodint[]{f}{x} > \alpha
        } \\
        &=& f^{-1}(]\alpha,\, \infty[)
    \end{eqnarray}
    
    Que es abierto débil, resulta ser una vecindad de $x_0$, totalmente contenida en $\compl{E} C$. Luego, $\compl{E} C \subseteq \adh(\compl{} C) $, y entonces $\compl{E}C$ es abierto débil, y $C$ es cerrado débil.
\end{itemize}
\end{proof}

\begin{thm}\label{teo:teo-2-}
Sean $E$, $F$ espacios de Banach y $T:E\to F$ lineal continua para la topología fuerte. Entonces 
\begin{equation}
    \deffuncs{T}
    {[E,\, \topdebilE ]}{[F,\, \topdebil{F}{F'}]}
\end{equation}

Es también continua (y lineal) para la topología fuerte.
\end{thm}

\begin{proof}
\begin{equation*}
    \deffuncs{\PHI_i}{X}{Y} \quad i \in I
\end{equation*}
\begin{equation*}
    \deffuncs{\psi}{Z}{X} \text{ es continua} \iff \deffuncs{\PHI_i \circ \psi}{Z}{Y}\quad \forall i \in I
\end{equation*}

$Z = E$, $\deffuncs{\PHI_i = f}{F}{\rr}$, ie,

$X = F$ $f \in F'$
$Y = \rr$
$\deffuncs{T}{[E,\, \topdebilE]}{[F,\, \topdebil{F}{F'}]}$ es continua $\iff$ $\deffunc{f \circ T}
{[E,\, \topdebilE]}{\rr}
{x}{\prodint[]{f}{Tx}}$ es continua.

Pero la aplicación $x \in E \mapsto \prodint[]{f}{Tx}$ es continua para todo $f \in F'$, pues $f$ y $T$ lo son, o bien, pues
\begin{eqnarray}
\abs{\prodint[]{f}{Tx}} &\leq& \norm[F']{f} \norm[F]{Tx} \\
&\leq& \norm[F']{f} \norm[]{T} \norm[E]{x}
\end{eqnarray}
Recíprocamente, supongamos que $\deffuncs{T}{[E,\, \topdebilE]}{[F,\, \topdebil{F}{F'}]}$ es continua y lineal. Luego, $\gr(T)$ es cerrado en $E \times F$ para la topología débil $\topdebil{E \times F}{E' \times F'}$. Pero, los cerrados débiles son cerrados del tipo fuertes, y entonces $\fr(T)$ es cerrado fuerte en $E \times F$ dotado de la norma $\norm[E]{\bullet} + \norm[F]{\bullet}$, 
y entonces, por el teo. del grafo cerrado, $T$ es continuo fuerte.
\end{proof}

\section{Espacio bidual}
$E$ espacio de Banach $\implies$ $E'$ es Banach $\implies$ $(E')' = E''$ es Banach,
dotado de la norma 
\begin{equation*}
    \norm[E']{\xi} = \sup_{f \inE', f \not = 0} \frac{\abs{\prodint[]{\xi}{f}}}{\norm[E']{f}}
\end{equation*}

\textbf{Afirmación}: $E$ se puede identificar con un sev de $E''$ a través de una funcional, denominada la \textbf{funcional de evaluación}, definida por:
\begin{equation*}
    \deffunc{J}
    {E}{E''}
    {x}{Jc}
\end{equation*}
tal que
\begin{equation*}
    \deffunc{Jx}
    {E'}{E}
    {\prodint[]{Jx}{f}}{\prodint[]{f}{x}}
\end{equation*}

$J$ es evidentemente inyectiva, pues si $Jx = 0$, entonces $\prodint[]{f}{x} = 0\ \forall f \in E'$,
y entonces $x=0$. Esto permite identificar $E$ con $J(E)$, sev de $E''$.

Además, $J$ es una isometría, pues
\begin{eqnarray*}
    \norm[E'']{Jx} 
    &=& \sup_{f \in E',\, f \not = 0} \frac{\abs{\prodint[]{Jx}{f}}}{\norm[E']{f}}\\
    &=& \sup_{f \in E',\, f \not = 0} \frac{\abs{\prodint[]{f}{x}}}{\norm[E']{f}} = \norm[E]{x}
\end{eqnarray*}

Así, $J(E)$ es cerrado fuerte en $E''$, y entonces $E$ se identifica con un sev cerrado de $E''$.

\begin{defn}
$E$  se dice \textbf{reflexivo} si $J(E) = E''$.
\end{defn}
