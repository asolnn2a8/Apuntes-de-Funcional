%!TEX root = ../main.tex

%%%%% NÚMERO DE LA CÁTEDRA Y FECHA
\renewcommand{\catnum}{11 (9 No Presencial)}% 6NP %numero de catedra
\renewcommand{\fecha}{21 de abril de 2020}
%%%%%%%%%%%%%%%%%%
%Encabezado
\NAM[
\begin{minipage}{0.6\textwidth}
]{\begin{minipage}{0.7\textwidth}}
\begin{flushleft}
\hspace*{-0.5cm}\textbf{Fecha:} \fecha.
\end{flushleft}
\end{minipage}

\begin{center}
\LARGE\textbf{Cátedra \catnum}
\end{center}

%Fin encabezado

\begin{thm}[Caracterización operadores acotados]\label{thm:caract-op-acot}
Sea $T: \cd(T) \subseteq E \to F$; $\adh \cd(T) = E$, $T$ cerrado

Son equivalentes:
\begin{ienumerate}
    \item $\cd(T) = E$
    \item $T$ es acotado
    \item $\cd(T\adj) = F'$
    \item $T\adj$ es acotado
\end{ienumerate}
Y en estas condiciones, $\norm[\cl(E,\ F)]{T} = \norm[\cl(F',\ E')]{T\adj}$ % TODO: Cambiarle a la L bonita
\end{thm}

Defina $G = \gr T$ y $L = E \times \{ 0\}$

Que son sev de $E$ cerrados

\textbf{Recuerdo:} Ejercicio \#4 (Lista \#3)

\begin{ienumerate}
    \item $\ker T \times \{ 0\} = G \cap L$
    \item $E \times \im T = G + L$
    \item $\{ 0 \} \times \ker T\ort = G\ort \cap L\ort$
    \item $\im T\adj \times F' = G\ort + L\ort$
\end{ienumerate}

\begin{enumerate}
    \item $\im T $ es cerrado  \\ 
    $\iff $ $E\times \im T$ es cerrado en $E \times F$ \\
    $\iff$ $G + L$ es cerrado en $E \times F$.
    \item $\im T\adj$ es cerrada \\
    $\iff$ $\im T\adj \times F'$ es cerrada en $E' \times F'$ \\
    $\iff$ $G\ORT + L\ORT$ es cerrada en $E' \times F'$ 
    \item $(\ker T )\ort = \im T\adj$ \\
    $\iff$ $(\ker T)\ort \times F' = \img T\adj \times F'$ \\ 
    $\iff$ $(G \cap L)\ort = G\ort + L\ort$
    \item $(\ker T\adj)\ort = \im T$ \\
    $\iff$ $E \times (\ker T \adj) \ort = E \times \im T$ \\
    $\iff$ $(G\ort \cap L\ort) \ort = G + L$
\end{enumerate}

Con esto, el teorema \#1 es simplemente una reescritura del teorema en la pág 2, guía \#2.

\begin{proof} Del Teorema \eqref{thm:caract-op-acot}:\\
Usemos el Teorema \#1 para probar que $(i)\iff (iii)$ 

Probemos primero que $(i) \implies (iii)$

Sabemos que $(\ker T\adj)\ort \underset{\text{(iv) en Teo. 1}}{=}  \im T = F$\\ $ \implies \ker T\adj = \{ 0 \}$ gracias a $(i)$, luego $T\adj$ es inyectivo.

Además $\im T\adj = (\ker T )\ort$ y entonces cerrada.

$(iii) \underset{\text{Por Teo, 1}}{\implies}  \im T$ es cerrada, pero además, por Teo. 1 parte $(iv)$\\ $\im T = (\ker T\adj)\ort$ = \{ 0 \}\ort = F $.

Demostremos que $(ii) \implies (iii)$

Si $T \adj y = 0$, entonces como $\norm[]{y} \leq c \norm[]{T\adj y} = 0$, sigue que $y = 0$. Así $T \adj$ es inyectivo.

Sea $(y_n)_n$ sucesión en $\im T\adj$ tal que $y_n \to y$ en $E'$

pdq $y \in \im T\adj$

Escribamos $y_n = T f_n$ con $f_n \in \cd(T\adj)$

Gracias a $(ii)$, $\forall n,\ m$ $$
\norm[]{f_n - f_m} \leq c \norm[]{y_n - y_m} \overset{n,\ m \to \infty}{\to} 0
$$
y entonces $(f_n)$ es de Cauchy en $F'$, y entonces $f_n \to f$ en $F'$. Con esto, $$
\left.
\begin{array}{r}
     f_n \in \cd(T\adj), f_n \to f \text{ en $F'$}\\
     y_n = T\adj f_n \to y  \text{ en $E'$}
\end{array}\right \}
\overset{\text{$T\adj$ es cerrado}}{\implies}  (f,\ y) \in \gr T\adj \text{, ie, } f \in \cd(T\adj),\ y = T\adj f
$$

Recapitulando:\\
$(i) \iff (iii)$ en Teo.2 V\\
$(ii) \implies (iii)$ en Teo.2 V

pdq $(iii) \implies (ii)$ o $(i) \implies (ii)$ en Teo. 2

Supongamos que $(i)$ en Teo. 2 es Verdadero.

y usaremos el Teorema de B-S al cjto. $$
G = \{ y \in \cd(T\adj) \subseteq F' |\ \norm[E']{T\adj y} \leq 1 \}
$$

Comencemos chequeando que es puntualmente acotado: Sea $f \in F$, $$
\sup_{y\in G} \abs{ \prodint[]{y}{f} }
\ \underset{\text{Pues $T$ es epiy}}{\overset{\text{para algún $x\in E$}}{=}}  \
\sup_{y\in G}\abs{ \prodint[]{y}{Tx} } 
= \sup_{y \in G} \abs{ \prodint[]{T\adj y}{x} }
\leq \underbrace{\sup_{y \in G} \norm[]{T\adj}}_{\leq 1}  \norm[]{x}
= \norm[]{x}
< \infty
$$

Gracias a B-S, $G$ está unif\mte acotada, ie, $\exists M \geq 0$ tal que $$
\norm[]{y} \leq M\quad \forall y \in G
$$

\textbf{Conclusión intermedia}

\begin{equation}\label{eqn:concl-int}
\left.
\begin{array}{r}
     y \in \cd(T\adj)\\
     \norm[E']{T\adj y} \leq 1
\end{array}\right \}
\implies \norm[F']{y} \leq M
\end{equation}

Luego, $\forall y \in \cd(T\adj)$ $$
\norm[]{y} \leq M \norm[]{T\adj y}
$$

En efecto, sea $z \in \cd(T\adj )$, cualquiera, y definamos $y = \frac{z}{\norm[]{T\adj z}}$. Luego, $$
\norm[]{z} = \norm[]{T\adj z} \norm[]{y}
$$

pues $\norm[]{T\adj y} = 1$

\end{proof}

\begin{proof} \textbf{(Del Teorema 3)}\\
Primero demostremos que $(i) \implies (ii)$ en Teo. 3

Usando Graf. Cerrado, (que se puede usar pues el esp. $[\cd(T)=E,\ \norm[]{\bullet}]$ es Banach)

Como $T$ es cerrado, es acotado o continuo

Demostremos que $(ii)\implies (iii)$ en Teorema 3.

pdq si $T$ es acot. cont. $\implies$ $\cd(T\adj) = F'$

Sea $y \in F'$. Miremos la app. $$
x\in E \mapsto \prodint[]{y}{Tx}
$$ Ahora bien, $$
\abs{ \prodint[]{y}{Tx} } 
\leq \norm[]{y} \norm[]{Tx}
\leq \underbrace{ \norm[]{y} \norm[]{T}}_{c} \norm[]{x} 
= c \norm[]{x}
$$ y entonces $y \in \cd(T\adj)$



\end{proof}

% $T: \cd(T) \subseteq E \to F$

% $G = \gr T$ y $L = E \times \{ 0\}$
