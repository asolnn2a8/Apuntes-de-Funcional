%!TEX root = ../main.tex

%%%%% NÚMERO DE LA CÁTEDRA Y FECHA
\renewcommand{\catnum}{\theNPclase \ No Presencial}% 6NP %numero de catedra
\renewcommand{\fecha}{5 de mayo de 2020}
%%%%%%%%%%%%%%%%%%
%Encabezado
\NAM[
\begin{minipage}{0.6\textwidth}
]{\begin{minipage}{0.7\textwidth}}
\begin{flushleft}
\hspace*{-0.5cm}\textbf{Fecha:} \fecha.
\end{flushleft}
\end{minipage}

\begin{center}
\LARGE\textbf{Cátedra \catnum}
\end{center}

%Fin encabezado

\begin{proof}
De la proposición \ref{prop:prop-3-conv-debil-acot}

\begin{ienumerate}
    \item $(x_n)_n \subseteq E$; $x_n \convdebil x$ en $E$-débil
    
    Probemos que $(x_n)$ es acotada, usando uno de los corolarios del teorema de B-St
    
    Sea $B=\{(x_n)\}$, $B$ es acotado si 
    \begin{equation}
        \bigcup_{n \in \nn} \{\prodint[]{f}{x_n} \} \text{ es acotado en $\rr$, $\forall f \in E'$}
    \end{equation}
    
    Pero como $f$ es continuo, $(\prodint[]{f}{x_n})_n$ es convergente y entonces es acotada.
    
    Además, razonando por dualidad, 
    
    \begin{eqnarray}
        \norm[E]{x} 
        &=& \sup_{f \in E'} \frac{\prodint[]{f}{x}}{\norm[E']{f}} \\
        &\underset{\text{sup se alcanza}}{=}& \frac{\prodint[]{f_0}{x}}{\norm[E']{f_0}} \\
        &=& \lim_{n \to \infty} \frac{\prodint[]{f_n}{x}}{\norm[E']{f_n}} \\
        &\leq& \norm[]{x_n}
    \end{eqnarray}
    
    Con esto,
    \begin{equation}
    \left.\begin{array}{l}
         \norm[]{x} = \lim_{n \to \infty} \frac{\prodint[]{f_n}{x}}{\norm[E']{f_n}}\\
         \frac{\prodint[]{f_n}{x}}{\norm[E']{f_n}} \leq \norm[]{x_n}
    \end{array}\right\} \implies \norm[]{x} \leq \liminf_{n\to\infty} \norm[]{x_n}
    \end{equation}
    
    \item pdq 
    
    \begin{equation}
    \left.\begin{array}{l}
         x_n \convdebil x \text{ en $E$-débil}\\
         f_n \to f \text{ en $E'$-fuerte}\\
    \end{array}\right\} \implies \prodint[]{f_n}{x_n} \to \prodint[]{f}{x}
    \end{equation}
    
    Veremos el módulo de la resta:
    
    \begin{eqnarray}
        |
        \prodint[]{f}{x} - \prodint[]{f_n}{x_n}
        |
        &\leq& 
        |
        \prodint[]{f}{x - x_n}
        | 
        + |
        \prodint[]{f - f_n}{x_n}
        | \\
        &\leq&
        |
        \prodint[]{f}{x - x_n}
        | 
        + \norm[]{f - f_n } \norm[]{x_n}\\
        &\leq&
        \underbrace{
        |
        \prodint[]{f}{x - x_n}
        |}_{\to 0}
        + \underbrace{
        M \norm[]{f - f_n }}_{\to 0}
    \end{eqnarray}
\end{ienumerate}
\end{proof}

\begin{prop}\label{prop:prop-4-base-de-vecindades}
Sea $x_0 \in E$, una base de vecindades abiertas débiles (para $\topdebilE$) son los conjuntos $V = V(x_0,\, \epsilon)$ de la forma 
\begin{equation}
    V = \{ x \in E | \abs{ \prodint[]{f_n}{x - x_0} } < \epsilon \ \forall i \in I \}
\end{equation}

Con $\epsilon > 0$ y $f_i \in E',\ \forall i \in I$, $I$ finito. O bien, equivalentemente

\begin{equation}
    V = \left\{ x \in E \Big| \sum_{i \in I} \abs{ \prodint[]{f_i}{x - x_0} } < \epsilon  \right\}
\end{equation}
\end{prop}

\begin{proof}
Los conjuntos $V$ son de la forma 
\begin{equation}
    V = \bigcap_{i \in I} f_i^{-1} \big( ]a_1 - \epsilon,\, a_i + \epsilon[ \big)
\end{equation}

con $a_i = f_i(x_0)$

Que es exactamente la forma de las bases de vecindades de $x_0 \in X$ en la topología $\sigma_X$ que hace continua una familia de funciones.
\end{proof}


\section{Estudio comparativo entre la topología débil $\topdebilE$ y la topología fuerte o de la norma $\topDeLaNorma$$}

Siempre se tiene que $\topdebilE \subseteq \topDeLaNorma$

\begin{thm}\label{teo:teo-1-dim-finit-top-coinciden}
Si $\dim E < + \infty$ ($E = \rr[N]$), entonces 
\begin{equation*}
    \topdebilE = \topDeLaNorma
\end{equation*}
\end{thm}

\begin{proof}
Suponga $E = <e_1,\, \ldots,\, e_N>$ con $\norm[E]{e_i} = 1$. Basta con probar que  si $V = B_{E}(x_0,\, r)$ es vecindad abierta de $x_0 \in E$ para la $\topDeLaNorma$, entonces $\exists U $ vecindad de $x_0$ en $\topdebilE$, $V \supseteq U$. Busquemos $U$ de la forma:
\begin{equation*}
    U = \left\{ 
    x \in E \Big| \sum_{i=1}^{m} \abs{ \prodint[]{f_i}{x - x_0} } < \epsilon
    \right\}
\end{equation*}

Donde $f_1,\, \ldots,\, f_m \in E'$ (buscamos $\epsilon$, $f_1,\, \ldots,\, f_m$ de manera que $U \subseteq V$). Para ello, escribamos que
\begin{eqnarray*}
    V &=& \{
    x \in E |
    \norm[E]{x - x_0} < r
    \} \\
    &=& \left\{ x \in E \Big|
    \norm[]{\sum_{i=1}^{N} (x_i - x_0) e_i} < r
    \right\} \\
    &=& \left\{ x \in E \Big|
    \norm[]{\sum_{i=1}^{N}
    \prodint[]{p_i}{x - x_0}
     e_i} < r
    \right\}
\end{eqnarray*}

Donde $p_i$ es la aplicación que a $x \in E \mapto x_i \in \rr$, el descomponer $x = \sum_{i=1}^{N} x_i e_i$.

Ahora bien, 
\begin{equation*}
    \norm[]{
    \sum_{i=1}^{N}\prodint[]{p_i}{x - x_0}e_i
    }
    \leq \sum_{i=1}^{N}\abs{\prodint[]{p_i}{x - x_0}},
\end{equation*}

y entonces, tomando $m = N $, $f_i = p_i$, $\epsilon = r$, se tiene que $U \subseteq V$
\end{proof}

Si $\dim E = + \infty$, entonces $\topdebilE  \subsetneq \topDeLaNorma$

\begin{ex}:\\
\begin{enumerate}
    \item $\dim E = + \infty$. Sea
    \begin{equation*}
        S = \conj{
        x \in E | \norm[]{x} = 1
        }
    \end{equation*}
    Es bien sabido que $S$ es cerrado fuerte. Sin embargo, no es cerrado débil, además
    \begin{equation*}
        \overline{S}^{\topdebilE} = \overline{B}_{\epsilon}(0,\, 1)= \conj{
        x \in E| \norm[]{x} \leq 1
        }
    \end{equation*}
    
    Resultado probado por Eduardo y Nicolás.
    
    \item Si $\dim E = + \infty$, entonces $B_{\epsilon} (0,\, 1)$ \textbf{no es} nunca abierta para $\topdebilE$. En efecto, sea $x_0 \in B_{E}(0,\, 1)$ y probemos que no existe $V$, vecindad de $x_0$ en $\topdebilE$ contenida en $B_{\epsilon} (0,\, 1)$
    
    En efecto, sea
    \begin{equation*}
        V = \conj{
        x \in E | \abs{\prodint[]{f_i}{x-x_o}} < \epsilon \ \forall i \in I \text{, con $I$ finito}
        }
    \end{equation*}
    
    Cin $f_i \in E'$, $\epsilon > 0$, una vecindad de $x_0$ en $\topdebilE$, y probemos que no es posible que $V \subseteq B(0,\, 1)$. Esto probaría que además, $(B_\epsilon(0,\, 1))^{\circ,\, \topdebilE} = \emptyset$.
    
    Admitiremos por un momento que 
    \begin{equation*}
        G = \bigcap_{i\in I} \ker(f_i) \not = \conj{0}
    \end{equation*}
    
    Fijemos $y \not = 0$, $y \in G$, y analicemos la recta 
    \begin{eqnarray}
    L:& x_0 + \lambda y,\ \lambda \in \rr
    \end{eqnarray}
    
    Vamos a afirmar que los $f_i$ son constantes a lo largo de $L$, pues $\prodint[]{f_i}{x_0 + y} = \prodint[]{f}{x_0}$ $\forall i \in I$. Y entonces $L \subseteq V$.
    
    ¿En $L$ hay elementos de norma $\geq 1$?
    
    La respuesta es sí, pues consideremos, la función real $h: \lambda \in \rr \mapsto \norm[]{x_0 + \lambda y} \in \rr$. Ahora bien, $h(0) = \norm[]{x_0} < 1$, $h(\lambda) \to \infty$ cuando $h \to \pm \infty$.
    
    Entonces, por continuidad, $\exists \Bar{\lambda}$, $h(\lambda) = M$, con $M$ real fijo cualquiera $\geq 1$. ($\im(h) = [x_0,\, \infty)$)
    
\end{enumerate}
\end{ex}

\begin{thm}
Sea $C \subseteq E$, $C$ convexo
\begin{equation*}
    \text{$C$ es cerrado débil $\iff$ $C$ es cerrado fuerte}
\end{equation*}
\end{thm}

\begin{proof}:\\
\begin{itemize}
    \item[($\Rightarrow$)] Directa
    
    \item[($\Leftarrow$)] Sea $x_0\in \mathcal{C}_{E} C$\footnote{$\mathcal{C}_{E} C$ denota el complemento con respecto a $E$ de $C$} (que es abierto fuerte)
    
    Separemos $x_0$ de $C$ (convexo cerrado fuerte) estrictamente.
    
    Luego $\exists f_0 \in E'$, $\exists \lambda \in \rr$ tal que
    \begin{equation*}
        \prodint[]{f_0}{x} < \alpha < \prodint[]{f_0}{x_0} \quad \forall x \in C
    \end{equation*}
    
    Luego, 
    \begin{equation*}
        U = f^{-1}\big( (\alpha,\, + \infty) \big),
    \end{equation*}
    
    Que es abierto débiñ, queda totalmente contenida en $\mathcal{C}_E C$. Así, $\mathcal{C}_E C$ es abierto débil.
\end{itemize}
\end{proof}