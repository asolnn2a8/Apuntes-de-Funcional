%!TEX root = ../main.tex

%%%%% NÚMERO DE LA CÁTEDRA Y FECHA
\renewcommand{\catnum}{8 (6 No Presencial)}% 6NP %numero de catedra
\renewcommand{\fecha}{13 de abril de 2020}
%%%%%%%%%%%%%%%%%%
%Encabezado
\NAM[
\begin{minipage}{0.6\textwidth}
]{\begin{minipage}{0.7\textwidth}}
\begin{flushleft}
\hspace*{-0.5cm}\textbf{Fecha:} \fecha.
\end{flushleft}
\end{minipage}

\begin{center}
\LARGE\textbf{Cátedra \catnum}
\end{center}

%Fin encabezado

\begin{defn}[Espacio ortogonal] \MarginNote{Espacio Ortogonal $M\ort$}
Sea $E$ ev Banach, $M$ sev de $E$. Se define:
$$M\ort \defeq \{ f \in E' | \prodint[E', E]{f}{x} = 0, \ \forall x \in M \} \quad \text{(Es cerrado en $E'$)}$$


Si $N$ sev de $E'$, se define
$$N\ort \defeq \{ x \in E | \prodint[E', E]{f}{x} = 0, \ \forall f \in N \} \quad \text{(Es cerrado en $E$)}$$
\end{defn}

\begin{prop}\MarginNote{Prop. de los ``ortogonal de ortogonales''} \\
\begin{ienumerate}
    \item ${(M\ort)}\ort = \adh M$
    \item ${(N\ort)}\ort \supseteq \adh N$
\end{ienumerate}
\end{prop}

Si el espacio es de Hilbert, el producto interno coincide con el producto escalar. De esta forma, esta definición coincide con la noción vista en espacio finito.

\begin{prop}
Sea $E$ un ev de Banach. $G$, $L$ sev de $E$ cerrados.
\begin{ienumerate}
    \item $G \cap L = {(G\ort + L\ort)}\ort$
    \item $G\ort \cap L\ort = {(G + L)}\ort$
\end{ienumerate}

\end{prop}

\begin{proof}: \\ 
\iitem[i] Sea $x \in G \cap L$ y probemos que $x \in {(G\ort + L\ort)}\ort$, ie, pdq $\prodint{f}{x} = 0 \forall f \in G\ort + L\ort$.\\ 
Pero $f \in G\ort + L\ort \iff f = f_1 + f_2$ , con $f_1 \in G\ort$ y $f_2 \in L\ort$, y entonces $\prodint{f_1}{x} = 0$ pues $x \in G$ y $\prodint{f_2}{x} = 0$ pues $x \in L$.\\
Por lo tanto $\prodint{f_1 + f_2}{x} = 0$. De esta forma, $x \in {(G\ort + L\ort)}\ort$

Inversa\mte, es claro que

$$G\ort \subseteq G\ort + L\ort \implies {(G\ort + L\ort)}\ort \subseteq {(G\ort)}\ort = \adh G = G$$

Pues $G$ es cerrado. Analoga\mte:

$${(G\ort + L\ort)}\ort \subseteq L $$

Entonces ${(G\ort + L\ort)}\ort \subseteq G \cap L $. Por lo tanto, ${(G\ort + L\ort)}\ort = G \cap L $. \\

\iitem[ii] pdq ${(G\ort \cap L\ort)}\ort = {(G + L)}\ort$\\
Sea $f \in G\ort \cap L\ort$ y probemos que $f \in {(G + L)}\ort$, ie, $$ \prodint{f}{x} = 0 \quad \forall x \in G + L $$

Pero, como $x = x_1 + x_2$ con $ x_1 \in G $, $ x_2 \in L $, sigue que $$ \prodint{f}{x_1} = 0 \quad \text{pues $f\in G\ort$}$$ $$ \prodint{f}{x_2} = 0 \quad \text{pues $f\in L\ort$}$$ Sumando, sigue que $$ \prodint{f}{x_1 + x_2} = 0 $$ y entonces $f \in (G + L)\ort$.

Inversa\mte: 
$$ G \subseteq G + L $$ tomando ortogonal 

$$\left. 
(G\ort + L)\ort \subseteq G\ort \atop
 (G\ort + L)\ort \subseteq L\ort 
 \right\} 
 \implies 
 (G\ort + L)\ort \subseteq G\ort \cap L\ort $$
 Concluyendo así que
 $$(G\ort + L)\ort = G\ort \cap L\ort $$
\end{proof}

$$G \cap L = (G\ort + L\ort)\ort$$

Tomando $\perp$: $$(G \cap L)\ort = {(
\underbrace{ G\ort + L\ort}_{\text{sev $E'$}}
)\ort}\ort
\supseteq \adh (G\ort + L\ort)$$
$$ G\ort \cap L\ort = (G + L)\ort $$
$$\overset{(\perp)}{\implies} (G\ort \cap L\ort)\ort = {(G + L)\ort}\ort = \adh (G + L)$$

Si $G$ y $L$ cerrados, ¿Será $G + L$ cerrado? La respuesta no es trivial, y la respuesta viene de un teorema (que no tiene tantas aplicaciones).

\section{Noción de Adjunto y Operadores No Acotados}

Vamos a poner énfasis en las diferencias en la dim. finita y dim infinita.

En dim. finita, lo interesante es que todo operador lineal es continua. La gran diferencia con la dim. infinita es que no todos los operadores lineales son continuos, y uno comienza a caracterizar la cont. en el caso de la dim. infinita.

En el caso de la dim finita se prueba que todos los op. lineales son continuos y acotados. De hecho, se hace una caracterización de que si es acotado, es continuo.

\begin{defn}\MarginNote{Op. Lineal \textbf{no acotado}}
Sea $E$, $F$ ev Banach. Un operador $T: \cd(T) \subseteq E \to F$ con $\cd(T)$ sev de $E$, se denomina operador (lineal) lineal no-acotado
\end{defn}

\begin{defn}\MarginNote{Op. Lineal \textbf{acotado}}
$T$ se dice \textbf{acotado} si $$\exists M \geq 0, \ \norm[F]{Tx} \leq M \norm[E]{x} \ \forall x \in \cd(T)$$
\end{defn}

Una función se le dice \textbf{no lineal} si es una función cualquiera. Y una función lineal es tmb. no lineal. Los matemáticos ``conviven'' con esta pequeña contradicción.

\begin{note}
Un operador acot. es tmb no-acotado, pero no hay que perder el sueño por esto.
\end{note}

En dim. finita, todo op. lineal se representa por una matriz.

\begin{exer}
Si $E = \rr[N]$, $F = \rr[M]$ y $T:\cd(T) = \rr[N] \to \rr[M]$, entonces $$(T \text{ es no-acotado} )  \iff (T \text{ es acotado} ) \iff (T \text{ es continua en $E$} )$$
\end{exer}

\begin{defn}
Un op. $T: \underset{\text{($\cd(T)$ sev de $E$)}}{\cd(T)} \subseteq E \to F$ se dice \textbf{cerrado} si $ \gr (T)$ es cerrado en $E \times F$
\end{defn}

Recordando que $$\gr T = \{ (x, \ Tx) | x \in \cd (T) \} \subseteq E \times F$$

\begin{note}: 
\begin{ienumerate}
    \item $T$ cerrado $\implies$ $\ker T$ es cerrado, pues $$ \ker T \times \{ 0\} = \gr T \cap (E \times \{ 0\}), \quad \text{que es cerrado}$$ 
    
    \item $T: \cd(T) \subseteq E \to F$ es cerrado $\iff$
    $$\left\{
    \begin{array}{l}
    {x_n \in \cd(T), \ x_n \to x \in E}\\
    {y_n = T x_n \to y_n} \\
    {\Bigl(ie, \ (x_n, \ y_n) \in \gr T, \ (x_n, \ y_n) \to (x, \ y) \Bigr)}
    \end{array}
    \right\} \implies x \in \cd(T) \wedge y = T x
    $$
    
    \item Para $T$ cerrado, \textbf{no} podemos usar el teo. del grafo cerrado, y concluir que $T$ es continua. (hay que evitar la tentación de ocupar este teorema y decir que $T$ es cont).
\end{ienumerate}
\end{note}