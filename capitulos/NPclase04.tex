%!TEX root = ../main.tex

%%%%% NÚMERO DE LA CÁTEDRA Y FECHA
\renewcommand{\catnum}{\theNPclase \ No Presencial}% 6NP %numero de catedra
\renewcommand{\fecha}{6 de abril de 2020}
%%%%%%%%%%%%%%%%%%
%Encabezado
\NAM[
\begin{minipage}{0.6\textwidth}
]{\begin{minipage}{0.7\textwidth}}
\begin{flushleft}
\hspace*{-0.5cm}\textbf{Fecha:} \fecha.
\end{flushleft}
\end{minipage}

\begin{center}
\LARGE\textbf{Cátedra \catnum}
\end{center}

%Fin encabezado

\section{Suplementario Topológico}

Sea $[E, \norm[]{\bullet}]$ un espacio de Banach.

\begin{defn}[Suplementario Topológico]
Sea $E$ ev y $G$ sev de $E$. Se dice que $L$, sev es suplementario de $G$ si $E = G \oplus L$ (ie, si $E = G + L \wedge G \cap L = \emptyset$)
\end{defn}

Sigue que $\forall z \in E$, este se descompone de manera única como $z = x + y$, $x \in G$, $y \in L$.

Todo $G$ sev de $E$ posee suplementario algebraico. Definiendo $x \sim y \iff (x - y) \in G$, obtenemos $$
\restr{E}{G} \defeq \{ \dot{a} | x \in E,\ \dot{x} = \{ y \in E | (x - y )\in G \}  \},
$$ donde $$
\dot{x} = \{ x \} + G
$$

$\restr{E}{G}$ es un sev. $\restr{E}{G}$ es, salvo identificación, un suplementario algebraico de $G$.

Identificación: $\Phi: \dot{x} \mapsto \Phi ( \dot{x} ) = \Bar{x}$, donde $\Bar{x}$ es un elemento escogido de $\dot{x}$.

$\forall x \in E$,
\begin{align*}
&x = \Phi (\dot{x}) + \underbrace{(x - \Phi (\dot{x}))}_{\in G}\\
\implies & E = G + \Phi ( \restr{E}{G} )
\end{align*}

Si $E$ es evn, $\restr{E}{G}$ es un evn con la norma $$
\norm[{E |}_{G}]{\dot{x}} = \inf_{x \in \dot{x}} \norm[E]{x}
$$ 

\begin{exer}
$E$ Banach y $G$ xerrado de $E$, entonces $[\restr{E}{G},\ \norm[{E|}_{G}]{\bullet}]$ es Banach.
\end{exer}

\begin{defn}
Sea $E$ Banach y $G$ sev cerrado. Diremos que $L$ es un suplementario topológico de $G$ si:
\begin{ienumerate}
    \item $L$ es cerrado
    \item $E = G\oplus L$
\end{ienumerate}
\end{defn}

\begin{rem}
Si $G$ y $L$ son suplementarios topológicos uno del otro, entonces $$
\deffunc{p_{G}}{E}{G}{x + y}{x}
$$ y $$
\deffunc{p_{L}}{E}{L}{x + y}{y}
$$ son continuas.

En efecto, sea $X = G \times L$ dotado de la norma $$
\norm[X]{(x,\ y)} = \norm[E]{x} + \norm[E]{y}
$$ y $$
\deffunc{T}{
[X,\ \norm[X]{\bullet}]
}{
[E = G + L,\ \norm[E]{\bullet}]
}{
(x,\ y)
}{
T(x,\ y) = x + y
}
$$ es biyectivo y continuo. Luego $T^{-1}$ es continuo, y entonces $\exists c \geq 0$ tal que: $$
\norm[X]{(x,\ y)} = \norm[E]{x} + \norm[E]{y} \leq c \norm[E]{x+y}
$$
\end{rem}

\NAM{\newpage}
\begin{ex}:\\
\begin{enumerate}[(1)]
    \item $E$ Banach y $G$ sev de dimensión finita, y entonces cerrado. 
    
    Su suplementario  algebraico es $E\restr{E}{G}$, que es Banach, pues $G$ es cerrado. Luego $\restr{E}{G}$ es cerrado y entonces es suplementario topológico.
    \item $E$ Banach y $G$ sev de $E$ de codimensión finita, ie $$
    \cod{G} \defeq \dim \restr{E}{G} < + \infty,
    $$ Luego $\restr{E}{G}$ es de dimensión finita, y entonces cerrado.
\end{enumerate}
\end{ex}

Resulta que $G = \restr{E}{L}$, y como $L$ es cerrado, $G$ también lo es.