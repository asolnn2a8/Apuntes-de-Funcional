%!TEX root = ../main.tex

%%%%% NÚMERO DE LA CÁTEDRA Y FECHA
\renewcommand{\catnum}{\theNPclase \ No Presencial}% 6NP %numero de catedra
\renewcommand{\fecha}{7 de abril de 2020}
%%%%%%%%%%%%%%%%%%
%Encabezado
\NAM[
\begin{minipage}{0.6\textwidth}
]{\begin{minipage}{0.7\textwidth}}
\begin{flushleft}
\hspace*{-0.5cm}\textbf{Fecha:} \fecha.
\end{flushleft}
\end{minipage}

\begin{center}
\LARGE\textbf{Cátedra \catnum}
\end{center}

%Fin encabezado

\begin{ex}:\\
\begin{enumerate}[(1)]
    \item $E$ Banach, $G$ sev cerrado de $E$, $\dim G < + \infty$. Con esto, $\restr{E}{G}$ es cerrado y entonces es suplementario topológico de $G$.
    \item $G$ sev de $E$ de codimensión finita, ie, $\dim \restr{E}{G} < + \infty$
    \item $E$ es un espacio de Hilbert. Todo sev $G$ de $E$ cerrado, posee suplementario topológico y es: $$
    L = G\ort = \{ x \in E | \prodint[]{x}{y} = 0 \quad \forall y \in G \}
    $$
    \item $E$ Banach, $f_1,\ \ldots,\ f_n \in E'$ y $G=\{ x \in E | \prodint[]{f_i}{x} = 0 \quad \forall i = 1,\ \ldots ,\ n \}$ (es cerrado y sev).
\end{enumerate}
\end{ex}

\begin{prop}
$$
\codim G = n
$$
\end{prop}

\begin{proof}
Demostraremos que $\exists x_1,\ \ldots,\ x_n \in E$ l.i. tales que $E = G \oplus \gen{x}{n}$.

Consideremos la aplicación $$
\deffunc{\varphi}{E}{\rr[n]}{x}{\varphi(x) = ( \prodint[]{f_1}{x},\ \ldots,\ \prodint[]{f_n}{x} )}
$$
$\varphi$ es sobreyectiva, pues sino lo fuera $\exists \ver{\alpha} \not= \ver{0} \in \rr[n]$ tal que $\ver{\alpha} \cdot ( \prodint[]{f_1}{x},\ \ldots,\ \prodint[]{f_n}{x} ) \quad \forall x \in E$, ie: 
% $$
% \sum_{i=1}^{n} \alpha_i \prodint[]{f_i}{x} = 0 \iff \prodint[]{\sum_{i=1}^{n} \alpha_i f_i}{x} = 0 \implies \sum_{i=1}^{n} \alpha_i f_i = 0 \implies \alpha_i = 0 \quad \forall i \contradic
% $$
$$
\begin{align*}
    \sum_{i=1}^{n} \alpha_i \prodint[]{f_i}{x} = 0 \iff \prodint[]{\sum_{i=1}^{n} \alpha_i f_i}{x} = 0 \implies \sum_{i=1}^{n} \alpha_i f_i = 0 \implies \alpha_i = 0 \quad \forall &i \\ &\contradic
\end{align*}
$$

Como $\varphi$ es sobreyectiva, tomando $e_i = (0,\ \ldots ,\ \underbrace{1}_{i},\ \ldots ,\ 0) \in \rr[n]$, $\exists x_i \in E$ tal que $$
\varphi (x_i) = e_i
$$

Así: $\prodint[]{f_j}{x_i} = \delta_{ij} \quad \forall i,\ j = 1,\ \ldots ,\ n$

Luego $E = $
\end{proof}